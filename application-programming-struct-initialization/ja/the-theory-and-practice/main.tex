\documentclass[a4j,12pt]{jreport}
\usepackage[dvipdfmx]{graphicx}
\usepackage{here}
\usepackage{listings}
\lstset{
    frame=single,
    numbers=left,
    tabsize=2,
    basicstyle=\ttfamily\footnotesize,
    breaklines=true,
}
%\usepackage{slashbox}
%\usepackage{subfigure}

\title{
\LARGE{アプリケーションプログラミングにおける\\構造体の初期化に関する論考}\\
\Large{〜理論と実践〜}
}
\author{市川恭佑\\\texttt{papers@mail.ebi-yade.com}}
\date{}

\begin{document}
\maketitle
\pagenumbering{roman}
\tableofcontents
\newpage
\listoffigures
\listoftables
\newpage
\pagenumbering{arabic}
\newpage

\chapter{はじめに}
\section{研究背景}
\section{本論文の構成}
本論文の構成を以下に示す。\\
第2章では、本論文の基礎知識を述べる。\\
第3章では、本論文の提案手法を述べる。\\
第4章では、関連研究を紹介する。\\
第5章では、実験方法、実験結果を述べる。\\
第6章では、実験結果に対する評価と考察を述べる。\\
第7章では、本論文のまとめを述べる。

\chapter{基礎知識}
%section
\chapter{提案手法}
\chapter{関連研究}
\chapter{実験}
\chapter{評価と考察}
\chapter{おわりに}
\chapter*{謝辞}
thanks!

\newpage
\renewcommand{\bibname}{参考文献}
\begin{thebibliography}{2}
% rfc template. search by number via: https://www.rfc-editor.org/search/rfc_search_detail.php
% \bibitem{rfc} , ``", , .
\end{thebibliography}

\newpage
\end{document}
