\documentclass[a4paper,12pt]{jsreport}
\usepackage[dvipdfmx]{graphicx}
\usepackage{here}
\usepackage{listings}
\lstset{
    frame=single,
    numbers=left,
    tabsize=2,
    basicstyle=\ttfamily\footnotesize,
    breaklines=true,
}
%\usepackage{slashbox}
%\usepackage{subfigure}

\title{
\LARGE{アプリケーションプログラミングにおける\\構造体の初期化に関する論考}\\
\Large{〜理論と実践〜}
}
\author{市川恭佑\\\texttt{papers@mail.ebi-yade.com}}
\date{}

\begin{document}
\maketitle
\pagenumbering{roman}
\tableofcontents
\newpage
\listoffigures
\listoftables
\newpage
\pagenumbering{arabic}
\newpage

\chapter{序論}
\section{研究背景}
近年,GoやRustなどの強力な型システムを持つ言語の登場や,VSCodeやIDEAなどのコード解析機能が充実したIDE,そしてAWSやGCPに代表されるパブリッククラウドの台頭により,アプリケーションプログラミングを取り巻く環境は大きく変化した.
以前は,Webアプリケーションのみならず,コマンドラインツールなどを含む多様なアプリケーションの開発において,Ruby on Railsなどの包括的なフレームワークに則ったプログラミングが一般的だった.
これに対し,言語とIDEの発達はプログラマの認知負荷を効果的に減らした.
つまり,一人ひとりが書き下し,把握できるコードの範囲は大きく広がった.
また,パブリッククラウドは従量課金制を採用し,コンピューティングリソースを必要に応じて即座に用意できる機会を提供した.
これにより,アプリケーションの実行環境は,垂直よりも水平にスケールすることが好まれる傾向が強まった.
\par
上記の変化により,従来の``アプリケーションプログラマが把握するコード量を減らし,垂直なスケーリングを前提とした''フレームワークは徐々にシェアを減らしている.
アプリケーションプログラマは,支配的なフレームワークの束縛から解放され,より自由度の高い設計でコードを書けるようになった.
\par
しかし,これは新たな課題を生み出した.
それは,個々のアプリケーションを作るプログラマ自身が,ほとんど全ての期間における処理の流れとデータの扱いに責任を持つ必要に迫られたことである(以前,これはフレームワーク側の責務であった).
ゆえに,異なるアプリケーションのうち,似た振る舞いをする部分のコードどうしが大きく異なることも少なくない.
もちろん,一部は自由度の高さの代償として捉えることもできるが,全てではない.
記述的特徴に差異があったとしても,共通の骨法に沿ってコードが設計されていることは,多角的な利益になる.
作者以外がコードを読む際の認知負荷を減らし,また作者自身にとっても記述時の認知負荷を減らすだけでなく,意図しない挙動を防ぐ効果が期待できる.

\section{研究目的}
本研究はモダンな開発環境を前提としたアプリケーションプログラミングにおける構造体の初期化について,理論的な枠組みの構築と,実践での確認を目的としている.
ここで言うアプリケーションとは,実行可能なバイナリを生成するためのプログラムを主とし,また抽象化のために切り出されたモジュールも含める.
完全に外部利用を想定して機能を絞ったライブラリは定義に含まれないが,応用の余地はあると考える.
\par
サンプルコードの言語は原則としてGoを用いるが,あくまで表層の記述的特徴ではなく,先進的な言語に共通する設計思想に裏付けられた傾向に注目する.
言語間の表層的な差異を吸収した理論を提供するが,設計思想の大きく異なる言語,特に動的型付け言語には適応が困難である.

\section{本論文の構成}
本論文の構成を以下に示す。\\
第2章では、本論文の基礎知識を述べる。\\
第3章では、本論文の提案手法を述べる。\\
第4章では、関連研究を紹介する。\\
第5章では、実験方法、実験結果を述べる。\\
第6章では、実験結果に対する評価と考察を述べる。\\
第7章では、本論文のまとめを述べる。

\chapter{基礎知識}
%section
\chapter{提案手法}
\chapter{関連研究}
\chapter{実験}
\chapter{評価と考察}
\chapter{おわりに}
\chapter*{謝辞}
thanks!

\newpage
\renewcommand{\bibname}{参考文献}
\begin{thebibliography}{2}
% rfc template. search by number via: https://www.rfc-editor.org/search/rfc_search_detail.php
% \bibitem{rfc} , ``", , .
\end{thebibliography}

\newpage
\end{document}
